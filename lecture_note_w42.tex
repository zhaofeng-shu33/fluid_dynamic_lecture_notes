
下面我们针对两组假设条件下的能量守恒积分方程进行化简:
第一组假设
\newglossaryentry{idealFluid}
{
  name=理想流体,
  description={表面力$T_n=-p \v{n}$}
}
\newglossaryentry{potentialField}
{
  name=有势场,
  description={存在标量函数$\Pi$, 使得$\v{f}=-\nabla \Pi$}
}
\newglossaryentry{steadyFlow}
{
  name=定常流,
  description={流体任意物理量$Q$只与空间坐标有关而与时间无关,满足$\frac{\partial Q}{\partial t}=0$}
}
\begin{itemize}
\item \gls{idealFluid},\glsdesc{idealFluid}
\item \gls{potentialField},\glsdesc{potentialField}
\item 绝热
\item \gls{steadyFlow},\glsdesc{steadyFlow}
\end{itemize}

在第一组假设下,能量守恒积分方程可化为
\begin{equation}\label{eq:42energy}
\oiint\limits_{\Sigma} \rho (e+\frac{1}{2}|\v{v}|^2)\v{v}\cdot\v{n} dA +
\iiint\limits_{D} \rho (\nabla \Pi)\cdot \v{v} d\tau 
+ \oiint\limits_{\Sigma} p \v{n}\cdot\v{v}dA=0
\end{equation}
我们尝试把上式第二项的体积分化为面积分,为此用到:
\begin{align*}
\rho (\nabla \Pi)\cdot \v{v} = & \nabla \cdot (\Pi \rho \v{v}) - \Pi \nabla \cdot(\rho \v{v})\\
= & \nabla \cdot (\Pi \rho \v{v}) \quad \text{定常流的连续性方程}
\end{align*}
因此我们得到
\begin{equation}
\oiint\limits_{\Sigma} \rho[\Pi + \frac{p}{\rho} + e + \frac{1}{2} |\v{v}|^2](\v{v}\cdot \v{n})dA=0
\end{equation}
上式中第一项$\Pi$表示体积势的贡献,第二项和第三项之和表示焓(内能与压力势之和),最后一项表示动能。

如果我们取$D$是一个流管,如图\ref{fig:412}所示,侧面积分为零,因此我们有
\begin{equation}
\oiint\limits_{A_1} \rho[\Pi + \frac{p}{\rho} + e + \frac{1}{2} |\v{v}|^2](\v{v}\cdot \v{n})dA=
-\oiint\limits_{A_2} \rho[\Pi + \frac{p}{\rho} + e + \frac{1}{2} |\v{v}|^2](\v{v}\cdot \v{n})dA
\end{equation}
假设面积$A_1,A_2$很小流动可以看作均匀的,从而
\begin{equation}
\rho[\Pi_1 + \frac{p_1}{\rho_1} + e_1 + \frac{1}{2} |\v{v_1}|^2](\v{v_1}\cdot \v{n_1})A_1=
-\rho[\Pi_2 + \frac{p_2}{\rho_2} + e_2 + \frac{1}{2} |\v{v_2}|^2](\v{v_2}\cdot \v{n_2})A_2
\end{equation}
代入微元流管的连续性方程
$\rho(\v{v}\cdot \v{n})A_1=-\rho(\v{v}\cdot \v{n})A_2$,我们得到
\begin{equation}
\Pi_1 + \frac{p_1}{\rho_1} + e_1 + \frac{1}{2} |\v{v_1}|^2 = \Pi_2 + \frac{p_2}{\rho_2} + e_2 + \frac{1}{2} |\v{v_2}|^2
\end{equation}
因此沿流线,我们得到第一组假设下的Bernoulli方程
\begin{equation}
\Pi + \frac{p}{\rho} + e + \frac{1}{2} |\v{v}|^2=c
\end{equation}
第二组假设
\newglossaryentry{uncompressibleCondition}
{
  name=不可压缩条件,
  description={流体微团体积不随时间变化,满足$\frac{D \rho}{D t}=0$}
}
\begin{itemize}
\item \gls{idealFluid}
\item \gls{potentialField}
\item \gls{uncompressibleCondition},\glsdesc{uncompressibleCondition}
\item \gls{steadyFlow}
\end{itemize}
在第二组假设下,由于体膨胀作功的部分为零,由热力学第一定律,内能的变化只与热交换有关,
因此在能量守恒方程中我们可以将内能与热交换项分离出来,从而得到\eqref{eq:42energy}式,但此时$e=0$,
类似的化简得到在第二组假设下的Bernoulli方程
\begin{equation}
\Pi + \frac{p}{\rho} + \frac{1}{2} |\v{v}|^2 = c
\end{equation}
进一步假设有势场是重力场,即$\Pi=gz$,$z$代表高度方向坐标,则得到Bernoulli方程的常用形式:
\begin{equation}
gz + \frac{p}{\rho} + \frac{1}{2} |\v{v}|^2 = c
\end{equation}

\begin{equation}
\end{equation}
