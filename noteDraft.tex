\documentclass{article}
\usepackage{xeCJK}
\usepackage{amsmath,esint,amssymb,amsthm}
\usepackage{hyperref}
\usepackage{tikz}
\usepackage[bottom]{footmisc}
\setCJKmainfont[AutoFakeBold]{SimSun}
\usepackage{bm}
\def\v#1{\overrightarrow{#1}}
\begin{document}
\title{笔记整理}
\author{赵丰}
\maketitle
\section{草稿}

理想流体运动的基本方程
理想流体应力张量$T_{ij}=-p\delta_{ij}$,因此动量方程\eqref{eq:52Momen}式化为:
\begin{equation}\label{eq:62EulerEq}
 \frac{\partial \v{v}}{\partial t}+\v{v}\cdot\nabla\v{v}=\v{f}-\frac{1}{\rho}\nabla\cdot p
\end{equation}
\eqref{eq:62EulerEq}式即欧拉方程,
相应的\eqref{eq:52Energy}化为
\begin{equation}\label{eq:62IdealEnergy}
 \frac{D(e+\frac{1}{2}|\v{v}|^2)}{Dt}=
 \v{f}\cdot\v{v}-\frac{1}{\rho}\nabla\cdot(p\v{v})+(\dot{q}+q_R) 
\end{equation}

综合连续性方程\eqref{eq:52Conti}式,\eqref{eq:62EulerEq},\eqref{eq:52Energy}
共5个方程,但未知数有$\v{v},e,p,\rho\,$ 6个,因此需补充热力学方程才能使方程组封闭。

实际上,我们可以通过动量方程\eqref{eq:62EulerEq}式得到动能的变化率:
\begin{equation}\label{eq:62KinematicEneryChangeRate}
\frac{D}{D t}(\frac{1}{2}|\v{v}|^2)=\v{f}\cdot \v{v}-\frac{1}{\rho}\v{v}\cdot\nabla p
\end{equation}
将\eqref{eq:62IdealEnergy}式与\eqref{eq:62KinematicEneryChangeRate}作差得:
\begin{align}
\frac{De}{Dt}= & -\frac{p}{\rho} \nabla \cdot \v{v} + \dot{q}+q_R ,\text{by }\eqref{eq:52ContiMaterial}\nonumber\\
=& \frac{p}{\rho^2}\frac{D\rho}{Dt} + \dot{q}+q_R \nonumber\\
=& -p\frac{D}{Dt}\left(\frac{1}{\rho}\right) + \dot{q}+q_R\label{eq:62DeDt1}
\end{align}
定义$i=e+\frac{p}{\rho}$为气体的焓,则可以得到
\begin{equation}
\frac{D i}{D t}=\dot{q}+q_R + \frac{1}{\rho} \frac{D p }{D t}
\end{equation}
对于绝热状态下的理想常比热完全气体,我们有
\begin{align}
\frac{D e}{D t}=& C_V \frac{D T}{D t},e=C_V T\nonumber\\
=& \frac{C_V}{R} \frac{D}{D t}\left(\frac{p}{\rho}\right),p=\rho RT\nonumber\\
=& \frac{1}{\gamma -1} \frac{D}{D t}\left(\frac{p}{\rho}\right),C_P-C_V=R,\frac{C_P}{C_V}=\gamma \label{eq:62DeDt2}
\end{align}
将\eqref{eq:62DeDt1}式去掉产热项,与\eqref{eq:62DeDt2}式结合可以得到
\begin{equation}\label{eq:62prho}
\frac{D}{D t}\left(\frac{p}{\rho^{\gamma}}\right)=0
\end{equation}
\eqref{eq:62prho}式即为对于气体补充的$\rho$和$p$的关系的热力学方程。

对于匀质不可压缩的液体,补充$\nabla \cdot \v{v}=0$的方程,此时\eqref{eq:52Conti}式恒成立,动力学方程与热力学方程解耦,
因此我们可以联立求解:
\begin{align}\notag
\nabla \cdot \v{v}=&0\\
 \frac{\partial \v{v}}{\partial t}+\v{v}\cdot\nabla\v{v} =&\v{f}-\frac{1}{\rho}\nabla\cdot p
\end{align}
得到$\v{v},p$再代入能量方程求其他参量。

下面考虑理想流体动力学偏微分方程的边界条件。一般的,对于不可穿透的壁面,流体的法向速度与壁面运动的法向速度相等,即满足:
\begin{equation}
(\v{v}_L\cdot \v{n})_{\Sigma}=(\v{v}_B\cdot \v{n})_{\Sigma}
\end{equation}
若$\Sigma$有曲面方程$F(x,y,z,t)=0$,则流体边界速度满足
\begin{equation}
\frac{\partial F}{\partial t} + \v{v_L} \cdot \nabla F=0
\end{equation}
此即\textbf{不可穿透条件}的一种提法;此外,我们还有无穷远条件等。

\begin{equation}
\end{equation}

对于球坐标,基矢量随坐标变量的变化规律为\cite{Del}:
\begin{align}
\frac{\partial \v{e_r}}{\partial \theta}=&\v{e_{\theta}},\,\,  \frac{\partial \v{e_r}}{\partial \varphi}=\v{e_{\varphi}}\sin\theta\\
\frac{\partial \v{e_{\theta}}}{\partial \theta}=&-\v{e_r},\,\,  \frac{\partial \v{e_{\theta}}}{\partial \varphi}=\v{e_{\varphi}}\cos\theta\\
\frac{\partial \v{e_{\varphi}}}{\partial \varphi}=&-(\v{e_r}\sin\theta+\v{e_{\theta}}\cos\theta)
\end{align}

球坐标系下的梯度算子表示为:
\begin{equation}
\nabla = \frac{\partial }{\partial r} \v{e_r} + \frac{1}{r} \frac{\partial}{\partial \theta} \v{e_{\theta}}+
\frac{1}{r\sin\theta}
\frac{\partial }{\partial \varphi}\v{e_{\varphi}}
\end{equation}

因此球坐标系下对于矢量$\v{v}=v_r\v{e_r}+v_{\theta}\v{e_{\theta}}+v_{\varphi}\v{e_{\varphi}}$的散度为:
\begin{align*}
&(\frac{\partial }{\partial r} \v{e_r} + \frac{1}{r} \frac{\partial}{\partial \theta} \v{e_{\theta}}+\frac{1}{r\sin\theta}
\frac{\partial }{\partial \varphi}\v{e_{\varphi}})\cdot (v_r\v{e_r}+v_{\theta}\v{e_{\theta}}+v_{\varphi}\v{e_{\varphi}}) =\\
&\frac{\partial v_r}{\partial r}+\frac{1}{r}\frac{\partial v_{\theta}}{\partial \theta}+\frac{1}{r\sin\theta}
\frac{\partial v_{\varphi}}{\partial \varphi} +\frac{v_r}{r}\frac{\partial \v{e_r}}{\partial \theta}\cdot \v{e_{\theta}}+
(v_r\frac{\partial \v{e_r}}{\partial \varphi}+v_{\theta}\frac{\partial \v{e_{\theta}}}{\partial \varphi}+
v_{\varphi}\frac{\partial \v{e_{\varphi}}}{\partial \varphi})\cdot \frac{\v{e_{\varphi}}}{r \sin\theta}\\
=&\frac{\partial v_r}{\partial r}+\frac{1}{r}\frac{\partial v_{\theta}}{\partial \theta}+\frac{1}{r\sin\theta}
\frac{\partial v_{\varphi}}{\partial \varphi} +\frac{v_r}{r}+
(v_r\sin\theta+v_{\theta}\cos\theta)\frac{1}{r \sin\theta}\\
=&\frac{\partial v_r}{\partial r}+\frac{1}{r}\frac{\partial v_{\theta}}{\partial \theta}+\frac{1}{r\sin\theta}
\frac{\partial v_{\varphi}}{\partial \varphi} +\frac{2v_r}{r}+\frac{v_{\theta}\cos\theta}{r \sin\theta}
\end{align*}
所以
\begin{equation}
\nabla \cdot \v{v}=\frac{1}{r^2}\frac{\partial (r^2 v_r)}{\partial r}+
\frac{1}{r\sin\theta}\frac{\partial (v_{\theta}\sin\theta)}{\partial \theta}+
\frac{1}{r\sin\theta}\frac{\partial v_{\varphi}}{\partial \varphi} 
\end{equation}
\begin{equation}
\end{equation}

\begin{thebibliography}{9}
\bibitem{mixedProduct} \href{https://en.wikipedia.org/wiki/Triple_product}{https://en.wikipedia.org/wiki/Triple\_product}
\bibitem{velocityGradient}\href{http://www.continuummechanics.org/velocitygradient.html}{http://www.continuummechanics.org/velocitygradient.html}
\bibitem{angularVelocityTensor}\href{https://en.wikipedia.org/wiki/Angular_velocity#Angular_velocity_tensor}{https://en.wikipedia.org/wiki/Angular\_velocity\#Angular\_velocity\_tensor}
\bibitem{CylindricalCoordinates}\href{https://en.wikipedia.org/wiki/Divergence#Cylindrical_coordinates}{https://en.wikipedia.org/wiki/Divergence\#Cylindrical\_coordinates}
\bibitem{CurlFormular}\href{https://en.wikipedia.org/wiki/Curl_(mathematics)}{https://en.wikipedia.org/wiki/Curl\_(mathematics)}
\bibitem{FundamentalSolution}\href{https://en.wikipedia.org/wiki/Fundamental_solution}{https://en.wikipedia.org/wiki/Fundamental\_solution}
\bibitem{GreenFunction}\href{https://en.wikipedia.org/wiki/Green\%27s_function#Green.27s_functions_for_the_Laplacian}{https://en.wikipedia.org/wiki/Green\%27s\_function\#Green.27s\_functions\_for\_the\_Laplacian}
\bibitem{Del}\href{https://en.wikipedia.org/wiki/Del_in_cylindrical_and_spherical_coordinates}{https://en.wikipedia.org/wiki/Del\_in\_cylindrical\_and\_spherical\_coordinates}
\end{thebibliography}
\end{document}