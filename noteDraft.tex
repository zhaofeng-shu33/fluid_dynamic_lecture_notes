\documentclass{article}
\usepackage{xeCJK}
\usepackage{amsmath,esint,amssymb,amsthm}
\usepackage{hyperref}
\usepackage{tikz}
\usepackage{arcs}
\usepackage[bottom]{footmisc}
\setCJKmainfont[AutoFakeBold]{SimSun}
\usepackage{bm}
\def\v#1{\overrightarrow{#1}}
\begin{document}
\title{笔记整理}
\author{赵丰}
\maketitle
\section{草稿}
理想不可压缩无旋流动方程

对于理想不可压缩无旋流动,存在速度势函数$\varphi$,并且$\varphi$在域内满足Laplace方程,附加一定的边界条件即可由下面的式子先求出速度势:
\begin{align}
\Delta \varphi=&0\nonumber\\
\frac{\partial F}{\partial t}+\nabla \varphi \cdot \nabla F=&0,\text{壁面条件} \nonumber\\
\nabla \varphi_{|R=\infty}=&\v{v_{\infty}},\text{无穷远条件}\label{eq:81IdealUnCompressibleIrrotationalEq}
\end{align}
解出$\varphi$后,求梯度得速度场,再代入CL方程\eqref{eq:72CLEquation}式中求压强$p$即可。

单连域速度场的唯一性定理

如果求解域$D$是单连通的,采用设速度作差法我们可以得到速度差函数(仍记为$\v{v}$)也满足\eqref{eq:81IdealUnCompressibleIrrotationalEq}式。
为说明$\v{v}=\v{0}$,考虑动能积分:
\begin{align*}
T=&\frac{1}{2}\rho \int_{D} |\v{v}|^2 d\tau\\
=&\frac{1}{2}\rho \int_{D} \nabla\varphi\cdot\nabla\varphi d\tau\\
=&\frac{1}{2}\rho \int_{D} \nabla\cdot(\varphi \nabla\varphi) d\tau,\nabla^2\varphi=0\\
=&\frac{1}{2}\rho \int_{A} (\v{n}\cdot \nabla \varphi)\varphi dA\\
=&\frac{1}{2}\rho \int_{A} \varphi\frac{\partial \varphi}{\partial n} dA
\end{align*}
因此,如果在$D$的边界$A$中逐点指定$\varphi$的值或者$\varphi$沿着边界面的法向导数的值,那么速度差函数在边界$A$中逐点有
$\varphi\frac{\partial \varphi}{\partial n}=0$,因此上式中动能为零,从而推出$\v{v}=\v{0}$,即我们得到了单连通域解的唯一性的条件为
如下三类:
\begin{itemize}
\item (Dirichlet 边界条件)在$A$上指定$\varphi$ 
\item (Neumann 边界条件)在$A$上指定$\frac{\partial \varphi}{\partial n}$ 
\item (混合边界条件)在$A_1$上指定$\frac{\partial \varphi}{\partial n}$ ,在$A_2$上指定$\varphi$,并且$A$可以分成$A_1$和$A_2$的不交并。
\end{itemize}

速度势函数的极值原理:
\begin{itemize}
\item 速度势函数$\varphi$不能在域内达到极大值或极小值。可以用反证法,假设在某点$\varphi$取到极大值,那么以该点为球心取一个半径很小的球,
有面积分$\oiint_{\Sigma} \frac{\partial \varphi}{\partial n}dA<0$,但由Gauss公式等式左边可化为$\int_{D} \nabla\cdot \v{v}d\tau=0$,矛盾。
%\item 速度的模$|\nabla \varphi|$不能在域内达到极大值
\item 在重力场中,压强不能在域内达到极小值。
对\eqref{eq:72CLEquation}式作用Laplace算子,得到$\Delta p=-\frac{1}{2} \Delta |\v{v}|^2$
\begin{align*}
\Delta \frac{1}{2}|\v{v}|^2=&\frac{1}{2}\nabla\cdot(\nabla |\v{v}|^2)\\
=&\nabla\cdot(\nabla\v{v}\cdot\v{v})\\
=&\Delta \v{v}\cdot\v{v}+\nabla\v{v}\cdot\nabla\v{v}
\end{align*}
$\Delta \v{v}=\nabla\cdot(\nabla \nabla \varphi)=\nabla (\nabla\cdot(\nabla \varphi))=\v{0}$
所以$\Delta p=-\nabla\v{v}\cdot\nabla\v{v}\leq 0$,对任一域内点的邻域球有:
\begin{equation}
\oiint_{\Sigma}\frac{\partial p}{\partial n} dA=\oiint_{\Sigma}\v{n}\cdot\nabla p dA=\iiint_{D} \Delta p d\tau \leq 0
\end{equation}
所以该点不可能取到极小值。
\end{itemize}


理想无旋不可压流场势函数三类基本解:
\begin{enumerate}
\item 对于均匀流场$\varphi=u_{\infty}x+v_{\infty}y+w_{\infty}z+c$
\item 对于点源诱导的流场,首先推导速度场,由球对称性有$\v{v}=v(R)\v{e_R}$,考虑通过半径为$R$的球的流量有:$2\pi R v(R)=Q$,因此
$v(R)=\frac{Q}{4\pi R^2}$,由$\frac{\partial \varphi}{\partial R}=v(R)$积分得$\varphi(R)=-\frac{Q}{4\pi R}+c$,称$Q>0$为点源,若$Q<0$则称为点汇
\item 偶极子,考虑两个相互靠的很近的源和汇,并定义$m=\displaystyle\lim_{\delta l \to 0}(Q\delta l)>0$,由叠加原理,两个源和汇在空间中产生的速度场为
(假设$-Q$在原点,$+Q$在$(\delta l,0,0)$:
\begin{align}\notag
\varphi=&\frac{Q}{4\pi \sqrt{x^2+y^2+z^2}}+\frac{-Q}{4\pi \sqrt{(x-\delta l)^2+y^2+z^2}}\\
=&\frac{-mx}{4\pi(\sqrt{x^2+y^2+z^2})^3}\label{eq:81Pole3}
\end{align}
\end{enumerate}

下面举一个应用基本解待定系数求解的例子,考虑圆球(圆心在原点,半径为$a$)绕流问题,
无穷远处来流为$\v{v_{\infty}}=U_{\infty}\v{i}$,
考虑速度势函数有$\varphi=U_{\infty} x-\frac{qx}{R^3}$的解,
其中$R=\sqrt{x^2+y^2+z^2}$,这相当于均匀流场的基本解与偶极子的基本解的叠加,
由于$\varphi$的奇点在球心,属于流场之外,且流场为单边域,故只需通过壁面条件确定系数$q$,
为此,采用球坐标系,并设$\theta$为空间一点$\v{r}$与$\v{i}$
的夹角(与一般球坐标$\theta$定义不同)。则$x=R\cos\theta$,所以
\begin{equation}
\varphi=\cos\theta(U_{\infty}R-\frac{q}{R^2})
\end{equation}
由壁面不可穿透性条件:$\frac{\partial \varphi}{\partial R}_{|R=a}=0$解出$q=-\frac{1}{2}a^3U_{\infty}$,所以
\begin{align*}
\varphi=&U_{\infty}R(1+\frac{a^3}{2R^3})\cos\theta\\
v_R=&\frac{\partial \varphi}{\partial R}=U_{\infty}(1-\frac{a^3}{R^3})\cos\theta\\
v_{\theta}=&\frac{1}{R}\frac{\partial \varphi}{\partial \theta}=-U_{\infty}(1+\frac{a^3}{R^3})\sin\theta
\end{align*}
由Bernoulli方程\eqref{eq:71LambBernoulli}式(不考虑有势力)可进一步解出$p$
如果考虑壁面$R=a$的速度和压力分布,$v_R=0$,$v_{\theta}=-\frac{3}{2}U_{\infty}\sin\theta$,
$p=p_{\infty}-\frac{\rho}{2}U_{\infty}^2(\frac{9}{4}\sin^2\theta-1)$
当$\theta=0$或$\pi$时$v_{\theta}=0$,此时速度为零,分别对应着后驻点和前驻点,压强为$p_0=p_{\infty}+\frac{1}{2}\rho U^2_{\infty}$
当$\theta=\frac{\pi}{2}$(或$\frac{3\pi}{2}$)时,此时速度达到最大,大小为$\frac{3}{2}v_{\infty}$,压强达到最小,
$p_{\textrm{min}}=p_{\infty}-\frac{5}{8}\rho U^2_{\infty}$。

针对平面不可压流,可引入流函数$\psi$的概念,有了流函数,速度场可写成:
\begin{equation}
\begin{cases}
u=& \frac{\partial \psi}{\partial y}\\
v=& -\frac{\partial \psi}{\partial x}
\end{cases}
\end{equation}

于是有
$\frac{\partial u}{\partial x}+\frac{\partial v}{\partial y}=0$,即不可压条件自动满足。若已知速度场$(u,v)$,可对$udy-vdx$做路径积分,
由Green公式,上面的微分形式环路积分为:
\begin{equation}
\oint_{l} udy-vdx=\iint_{A} (\frac{\partial u}{\partial x}+\frac{\partial v}{\partial y})dxdy=0
\end{equation}
因此对不可压平面流场,流函数总是存在的,可写成
\begin{equation}\label{eq:81streamFun}
\psi(x,y)=\int_{(x_0,y_0)}^{(x,y)} (udy-vdx)
\end{equation}

流函数具有如下的性质:
\begin{itemize}
\item 平面流线是流函数的等值线,因为
$\frac{dx}{u}=\frac{dy}{v}\Leftrightarrow udy-vdx=0 \Leftrightarrow d\psi=0 \Leftrightarrow \psi=c$
%$\psi(M_0)-\psi(M)$ ?
\item 通过某截曲线\overarc{$M_0M$}的体积流量等于$\psi(M)-\psi(M_0)$,因为
$\int_{M_0}^M (\v{v}\cdot \v{n})ds=\int_{M_0}^M (\v{v}\cdot (dy,-dx))=\int_{M_0}^M (udy-vdx)$,于是由\eqref{eq:81streamFun}式可知。
\item 若等势线(速度势函数的等值线)存在,则流线与等势线正交。因为前者的法方向为:$(-v,u)$,后者的法方向为$(u,v)$。
\end{itemize}

对于平面无旋流动,由$\nabla\times \v{v}=0$ 可推出 $\psi$适合平面Laplace方程,附加适当的边界条件,对于单连域,$\psi$
满足
\begin{align}
\frac{\partial^2 \psi}{\partial x^2}+\frac{\partial^2 \psi}{\partial y^2}=&0\nonumber\\
(\frac{\partial \psi}{\partial y},\frac{\partial \psi}{\partial x})=&(u_{\infty},-v_{\infty}),\text{来流条件} \nonumber\\
\psi=&c,\text{物面不可穿透条件}
\end{align}

平面流场流函数三类基本解:
我们考虑满足$\Delta \psi=0$,
\begin{enumerate}
\item 对于均匀流场$\psi=u_{\infty}y-v_{\infty}x+c$
\item 对于点源诱导的流场,首先推导速度场,类似中的推导我们有$2\pi R v(R)=Q,\v{v}=v(R)\v{e_R}$,在极坐标系中,
$v(R)=\frac{1}{R}\frac{\partial \psi}{\partial \theta}$,所以$\psi=\frac{Q}{2\pi}\theta+c$,变换到直角坐标系中即为
\begin{equation}
\psi=\frac{Q}{2\pi}\arctan\frac{y}{x}+c
\end{equation}
\item 偶极子,类似\eqref{eq:81Pole3}式的推导,对$\arctan\frac{y}{x}$对$x$求导得到偶极子诱导的势函数为
\begin{equation}
\psi=\frac{-m}{2\pi}\frac{y}{x^2+y^2}+c
\end{equation}
\end{enumerate}
\begin{equation}
\end{equation}
\begin{equation}
\end{equation}

\begin{thebibliography}{99}
\bibitem{mixedProduct} \href{https://en.wikipedia.org/wiki/Triple_product}{https://en.wikipedia.org/wiki/Triple\_product}
\bibitem{velocityGradient}\href{http://www.continuummechanics.org/velocitygradient.html}{http://www.continuummechanics.org/velocitygradient.html}
\bibitem{angularVelocityTensor}\href{https://en.wikipedia.org/wiki/Angular_velocity#Angular_velocity_tensor}{https://en.wikipedia.org/wiki/Angular\_velocity\#Angular\_velocity\_tensor}
\bibitem{CylindricalCoordinates}\href{https://en.wikipedia.org/wiki/Divergence#Cylindrical_coordinates}{https://en.wikipedia.org/wiki/Divergence\#Cylindrical\_coordinates}
\bibitem{CurlFormular}\href{https://en.wikipedia.org/wiki/Curl_(mathematics)}{https://en.wikipedia.org/wiki/Curl\_(mathematics)}
\bibitem{FundamentalSolution}\href{https://en.wikipedia.org/wiki/Fundamental_solution}{https://en.wikipedia.org/wiki/Fundamental\_solution}
\bibitem{GreenFunction}\href{https://en.wikipedia.org/wiki/Green\%27s_function#Green.27s_functions_for_the_Laplacian}{https://en.wikipedia.org/wiki/Green\%27s\_function\#Green.27s\_functions\_for\_the\_Laplacian}
\bibitem{Del}\href{https://en.wikipedia.org/wiki/Del_in_cylindrical_and_spherical_coordinates}{https://en.wikipedia.org/wiki/Del\_in\_cylindrical\_and\_spherical\_coordinates}
\bibitem{CEEquation}\href{https://en.wikipedia.org/wiki/Cauchy\%E2\%80\%93Euler_equation}{https://en.wikipedia.org/wiki/Cauchy\%E2\%80\%93Euler\_equation}
\end{thebibliography}
\end{document}