牛顿流体的本构方程

式\eqref{eq:52TSG}给出了牛顿流体的本构方程,记$\lambda=\mu'-\frac{2}{3}\mu$,表示体膨胀系数。
\eqref{eq:52TSG}写为
\begin{equation}
T_{ij}=-p\delta_{ij}+\lambda S_{kk} \delta_{ij} +2\mu S_{ij}
\end{equation}
上式中第一项为流体静止时的应力,后两项之和为黏性应力。定义流体的力学压强
\begin{equation}
\bar{p}=-\frac{1}{3}T_{ii}
       =p-\mu'S_{kk}
\end{equation}
可以看出若第二黏性应力$\mu'=0$,
则力学压强$\bar{p}$与热力学压强$p$相等。

NS方程--牛顿流体的运动方程

式\eqref{eq:71generalNS}给出了$\mu'=0$时的Navier-Stokes方程,这里我们考虑流体不可压的情形,即$\nabla\cdot \v{v}=0$,此时
\eqref{eq:71generalNS}化为
\begin{equation}\label{eq:102unCompressibleNS}
\frac{\partial \v{v}}{\partial t}+\v{v}\cdot\nabla\v{v}=\v{f}-\frac{\nabla p}{\rho}+\nu \nabla^2 \v{v}
\end{equation}
其中$\nu=\frac{\mu}{\rho}$被称为运动黏性系数。

黏性流体耗散项的影响

在第\ref{sec:62}小节中,针对理想流体我们推导了流体内能、焓的变化。
%熵的变化没懂,不知道怎么写
对于一般黏性流体的情形,我们将\eqref{eq:52Energy}式表示的能量方程的产热项和热交换项合并为$Q$于是可得:
\begin{equation}
\frac{D}{Dt}(e+\frac{1}{2}|\v{v}|^2)=\v{f}\cdot\v{v}+\frac{1}{\rho}\cdot (\bm{T}\cdot\v{v})+Q
\end{equation}
对\eqref{eq:52momen}式表示的动量方程两边同时点乘$\v{v}$可得:
\begin{equation}
\frac{D}{Dt}(\frac{1}{2}|\v{v}|^2)=\v{f}\cdot\v{v}+\frac{1}{\rho}\v{v}\cdot(\nabla \cdot \bm{T})
\end{equation}
将上面两式相减,得到内能的变化为
\begin{equation}
\frac{D e}{Dt}=\frac{1}{\rho}(\nabla\cdot (\bm{T}\cdot\v{v})-\v{v}\cdot(\nabla \cdot \bm{T}))+Q
\end{equation}
化简
\begin{align}\notag
\nabla\cdot (\bm{T}\cdot\v{v}) = & (\frac{\partial}{\partial x_k}\v{e_k})\cdot (T_{ij}v_j\v{e_i})\\
= & \frac{\partial T_{ij}}{\partial x_i} v_j + \frac{\partial v_j}{\partial x_i} T_{ij}\label{eq:102nTv}
\end{align}
利用\eqref{eq:52TijUn}式
\begin{align}\notag
\v{v}\cdot(\nabla \cdot \bm{T}) =& (v_k \v{e_k}) \cdot (\frac{\partial T_{ij}}{\partial x_i}\v{e_j})\\
=& v_j \frac{\partial T_{ij}}{\partial x_i}\label{eq:102vnT}
\end{align}
从\eqref{eq:102nTv},\eqref{eq:102vnT}两式得到:
\begin{align*}
\nabla\cdot (\bm{T}\cdot\v{v})-\v{v}\cdot(\nabla \cdot \bm{T}) & = \frac{\partial v_j}{\partial x_i} T_{ij} \\
& = \frac{\partial v_j}{\partial x_i} (-p\delta_{ij}+\lambda S_{kk} \delta_{ij} +2\mu S_{ij}) \\
& = -p S_{ii} +\lambda S_{ii}^2 + 2\mu S_{ij}S_{ij}
\end{align*}
所以
\begin{equation}\label{eq:102etIncomplete}
\frac{D e}{Dt}= -\frac{p}{\rho} S_{ii} + \Phi + Q
\end{equation}
其中
\begin{equation}
\Phi=\frac{1}{\rho}(\lambda S_{ii}^2 + 2\mu S_{ij}S_{ij})
\end{equation}
由上式,$\Phi\leq 0$,$\Phi$被称为耗散项,其作用是使流体微元的内能增加。
由连续性方程\eqref{eq:52ContiMaterial}式,\eqref{eq:102etIncomplete}式可改写为
\begin{align}\notag
\frac{D e}{Dt} =& \frac{p}{\rho^2} \frac{D\rho}{Dt} + \Phi + Q\\
= & -p  \frac{D}{Dt}\frac{1}{\rho} + \Phi + Q
\end{align}
上式与理想流体的内能变化\eqref{eq:62DeDt1}式对比可看出,二者的区别是是否有耗散项$\Phi$。
类似\eqref{eq:62Enthalpy}式,我们可推出焓的变化公式
\begin{equation}
\frac{D i}{D t} = \frac{1}{\rho} \frac{D p }{D t}+ \Phi + Q
\end{equation}

不可压牛顿型流体的封闭方程组与定解条件
利用\eqref{eq:102unCompressibleNS}式的结果,对于四个未知数$\v{v},p$,有如下4个标量偏微分方程组:
\begin{equation}\label{eq:102ClosedNSEqSystem}
\begin{cases}
\frac{\partial \v{v}}{\partial t}+\v{v}\cdot\nabla\v{v}&=\v{f}-\frac{\nabla p}{\rho}+\nu \nabla^2 \v{v}\\
\nabla\cdot\v{v}&=0
\end{cases}
\end{equation}
求解该偏微分方程组需要给出初始条件,即$t=0$时刻速度场的分布$\v{v}=\v{v_0}$,利用不可压条件,对
\eqref{eq:102unCompressibleNS}式两边取散度,
可求出$t=0$时
压力场满足Poisson方程
\begin{equation}
\nabla\cdot(\v{v}\cdot\nabla\v{v})=-\frac{\rho}\nabla^2 p +\nabla\cdot \v{f}
\end{equation}
其中用到了
\begin{align*}
\nabla\cdot(\nabla^2 \v{v}) = & (\frac{\partial}{\partial x_k}\v{e_k})\cdot (\frac{\partial^2 v_i}{\partial x_j^2}\v{e_i})\\
= & \frac{\partial^3 v_i}{\partial x_j^2\partial x_i} \\
= & 0
\end{align*}
求解\eqref{eq:102ClosedNSEqSystem}式表示的方程组还需要适当的边界条件,具体有
\begin{itemize}
\item 固壁无滑移条件:$\v{v}=\v{v_b}$
\item 界面条件
    \begin{itemize}
    \item 界面不可穿透条件$\frac{\partial F}{\partial t}+\nabla \varphi \cdot \nabla F=0$
    \item 无穷远速度条件
    \item 压力条件,忽略表面张力的影响,一般取液体自由面的压力等于大气压与液面切应力为零。
    \end{itemize}
\end{itemize}
